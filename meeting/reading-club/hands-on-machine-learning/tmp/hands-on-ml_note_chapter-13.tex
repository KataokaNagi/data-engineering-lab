\hypertarget{ux7b2c13ux7ae0-ux5c0eux5165}{%
\section{第13章 導入}\label{ux7b2c13ux7ae0-ux5c0eux5165}}

\hypertarget{tensorflowux306eux30c7ux30fcux30bfapi}{%
\subsubsection{TensorFlowのデータAPI}\label{tensorflowux306eux30c7ux30fcux30bfapi}}

\begin{itemize}
\tightlist
\item
  DNNの大規模なデータセットに対応
\item
  データの入手元と変換方法をのみを指示
\item
  以下はTF(Tensorflow)が代行

  \begin{itemize}
  \tightlist
  \item
    マルチスレッド管理
  \item
    キューイング
  \item
    バッチへの分割
  \item
    プリフェッチ
  \end{itemize}
\end{itemize}

\hypertarget{ux30c7ux30fcux30bfapiux3067ux306eux5165ux529bux5f62ux5f0f}{%
\subsubsection{データAPIでの入力形式}\label{ux30c7ux30fcux30bfapiux3067ux306eux5165ux529bux5f62ux5f0f}}

\begin{itemize}
\tightlist
\item
  拡張前

  \begin{itemize}
  \tightlist
  \item
    CSVなどのテキストファイル
  \item
    レコードが固定長のバイナリファイル
  \item
    レコードが可変長のTFRecord形式

    \begin{itemize}
    \tightlist
    \item
      柔軟で効率が良い
    \item
      通常はプロトコルバッファを格納(オープンソースのバイナリフォーマット)
    \end{itemize}
  \end{itemize}
\item
  その他、BigQueryなどに拡張可能
\end{itemize}

\hypertarget{ux524dux51e6ux7406}{%
\subsubsection{前処理}\label{ux524dux51e6ux7406}}

\begin{itemize}
\tightlist
\item
  主な対象

  \begin{itemize}
  \tightlist
  \item
    スケールの異なる数値フィールド群
  \item
    テキスト特徴量
  \item
    カテゴリ特徴量
  \end{itemize}
\item
  主な対処

  \begin{itemize}
  \tightlist
  \item
    正規化
  \item
    ワンホットエンコーディング
  \item
    バッグオブワーズエンコーディング
  \item
    埋め込み(embedding)(後述)
  \end{itemize}
\end{itemize}

\hypertarget{ux4fbfux5229ux306atfux30e9ux30a4ux30d6ux30e9ux30ea}{%
\subsubsection{便利なTFライブラリ}\label{ux4fbfux5229ux306atfux30e9ux30a4ux30d6ux30e9ux30ea}}

\begin{itemize}
\tightlist
\item
  TF Transform (tf.Transform)

  \begin{itemize}
  \tightlist
  \item
    訓練前:訓練セット全体を高速なバッチモードで前処理
  \item
    訓練後:TF関数として訓練済みモデルに組み込んで前処理

    \begin{itemize}
    \tightlist
    \item
      本番環境で新たなインスタンスをその場で処理可能
    \end{itemize}
  \end{itemize}
\item
  TF Datasets(TFDS)

  \begin{itemize}
  \tightlist
  \item
    大規模なデータセットがダウンロード可能な関数を提供
  \item
    データAPIで上記関数を操作可能なデータセットオブジェクトを提供
  \end{itemize}
\end{itemize}

\hypertarget{ux30c7ux30fcux30bfapi}{%
\section{13.1 データAPI}\label{ux30c7ux30fcux30bfapi}}

\hypertarget{ux30c7ux30fcux30bfux30bbux30c3ux30c8}{%
\subsubsection{データセット}\label{ux30c7ux30fcux30bfux30bbux30c3ux30c8}}

\begin{itemize}
\tightlist
\item
  ディスクなどから少しずつ読み出していくデータ要素のシーケンス
\item
  例:from\_tensor\_slices(X)

  \begin{itemize}
  \tightlist
  \item
    テンソルXの第1次元のスライスであるデータセットを返す
  \item
    つまり、tfのスライスをDatasetのスライスに変換?
  \end{itemize}
\end{itemize}

\hypertarget{ux5909ux63dbux306eux9023ux9396}{%
\subsection{13.1.1 変換の連鎖}\label{ux5909ux63dbux306eux9023ux9396}}

\hypertarget{ux5225ux306eux30c7ux30fcux30bfux30bbux30c3ux30c8ux306bux5909ux63dbux3059ux308bux30e1ux30bdux30c3ux30c9-14}{%
\subsubsection{別のデータセットに変換するメソッド
1/4}\label{ux5225ux306eux30c7ux30fcux30bfux30bbux30c3ux30c8ux306bux5909ux63dbux3059ux308bux30e1ux30bdux30c3ux30c9-14}}

\begin{itemize}
\tightlist
\item
  repeat(n)

  \begin{itemize}
  \tightlist
  \item
    同じスライスをn回連結
  \item
    コピーではない(高速かつ小容量?)
  \end{itemize}
\item
  batch(n)

  \begin{itemize}
  \tightlist
  \item
    スライスをn個ずつのスライスに分割
  \item
    最後のバッチがn個未満の場合、引数にdrop\_remainder=Trueで削除可能
  \end{itemize}
\end{itemize}

\hypertarget{ux5225ux306eux30c7ux30fcux30bfux30bbux30c3ux30c8ux306bux5909ux63dbux3059ux308bux30e1ux30bdux30c3ux30c9-24}{%
\subsubsection{別のデータセットに変換するメソッド
2/4}\label{ux5225ux306eux30c7ux30fcux30bfux30bbux30c3ux30c8ux306bux5909ux63dbux3059ux308bux30e1ux30bdux30c3ux30c9-24}}

\begin{itemize}
\tightlist
\item
  \href{https://qiita.com/conf8o/items/0cb02bc504b51af09099}{map}(lambda
  x: xの式)

  \begin{itemize}
  \tightlist
  \item
    要素をラムダ式で柔軟に変換するなど
  \item
    TF関数に変換可能なラムダ式のみ対応(12章)
  \item
    \href{https://tensorflow.classcat.com/2019/03/23/tf20-alpha-guide-data-performance/}{引数num\_parallel\_calls=tf.data.experimental.AUTOTUNEで並列・高速化}
  \end{itemize}
\end{itemize}

\hypertarget{ux5225ux306eux30c7ux30fcux30bfux30bbux30c3ux30c8ux306bux5909ux63dbux3059ux308bux30e1ux30bdux30c3ux30c9-34}{%
\subsubsection{別のデータセットに変換するメソッド
3/4}\label{ux5225ux306eux30c7ux30fcux30bfux30bbux30c3ux30c8ux306bux5909ux63dbux3059ux308bux30e1ux30bdux30c3ux30c9-34}}

\begin{itemize}
\tightlist
\item
  unbach()

  \begin{itemize}
  \tightlist
  \item
    bach()で作ったデータセットを解体
  \item
    試験段階のメソッドで、collabではエラー
  \end{itemize}
\item
  apply()

  \begin{itemize}
  \tightlist
  \item
    {[}{]}で囲まれたデータセット全体に適用
  \item
    引数にdataset.unbach()などを用いる
  \end{itemize}
\end{itemize}

\hypertarget{ux5225ux306eux30c7ux30fcux30bfux30bbux30c3ux30c8ux306bux5909ux63dbux3059ux308bux30e1ux30bdux30c3ux30c9-44}{%
\subsubsection{別のデータセットに変換するメソッド
4/4}\label{ux5225ux306eux30c7ux30fcux30bfux30bbux30c3ux30c8ux306bux5909ux63dbux3059ux308bux30e1ux30bdux30c3ux30c9-44}}

\begin{itemize}
\tightlist
\item
  filter(lambda x: xを用いたbool演算)

  \begin{itemize}
  \tightlist
  \item
    ラムダ式がtrueの要素xのみを用いる
  \end{itemize}
\item
  take(n)

  \begin{itemize}
  \tightlist
  \item
    戦闘からn個の要素を用いる
  \end{itemize}
\end{itemize}

\hypertarget{ux30c7ux30fcux30bfux306eux30b7ux30e3ux30c3ux30d5ux30eb}{%
\subsection{13.1.2
データのシャッフル}\label{ux30c7ux30fcux30bfux306eux30b7ux30e3ux30c3ux30d5ux30eb}}

\hypertarget{datasetux306eux30b7ux30e3ux30c3ux30d5ux30eb}{%
\subsubsection{Datasetのシャッフル}\label{datasetux306eux30b7ux30e3ux30c3ux30d5ux30eb}}

\begin{itemize}
\tightlist
\item
  shuffle(buffer\_size=n, seed=m)

  \begin{itemize}
  \tightlist
  \item
    要素のシャッフル
  \item
    独立同分布(同一の分布から独立に抽出された状態)でよく機能する勾配降下法などで使用(4章)
  \item
    大規模なデータからサイズnのバッファを経由してシャッフル
  \item
    大規模データセットに見合った大きいバッファサイズが必要
  \item
    RAMの容量に注意
  \item
    shuffle()後にrepeat(3)しても、3回とも異なるリピートとなる

    \begin{itemize}
    \tightlist
    \item
      引数reshuffle\_each\_iteration=Falseで同じリピートにできる
    \end{itemize}
  \end{itemize}
\end{itemize}

\hypertarget{ux4fbfux5229ux306aux30b7ux30e3ux30c3ux30d5ux30eb}{%
\subsubsection{便利なシャッフル}\label{ux4fbfux5229ux306aux30b7ux30e3ux30c3ux30d5ux30eb}}

\begin{itemize}
\tightlist
\item
  list\_files(DATA\_FILE\_PATH, seed=n)

  \begin{itemize}
  \tightlist
  \item
    データをシャッフルしてロード
  \item
    引数suffle=Falseも指定可能
  \end{itemize}
\item
  interleave(lambda file\_name: ラムダ式, cycle\_length=n)

  \begin{itemize}
  \tightlist
  \item
    n個のファイルを同時に無作為に読み出す
  \item
    n個のデータセットをもつ1個のデータセットを作成
  \item
    n個のデータセット間の同じ長さの部分が互い違いになる

    \begin{itemize}
    \tightlist
    \item
      \href{https://tensorflow.classcat.com/2019/03/23/tf20-alpha-guide-data-performance/}{引数num\_parallel\_calls=tf.data.experimental.AUTOTUNEで初めて並列化する}
    \end{itemize}
  \end{itemize}
\item
  コードでは、ファイルパスが尽きるまで5個ずつ取り出しインターリーブしている
\end{itemize}

\hypertarget{ux30c7ux30fcux30bfux306eux524dux51e6ux7406}{%
\subsection{13.1.3
データの前処理}\label{ux30c7ux30fcux30bfux306eux524dux51e6ux7406}}

\hypertarget{csvux30c7ux30fcux30bfux306eux30b9ux30b1ux30fcux30eaux30f3ux30b0}{%
\subsubsection{csvデータのスケーリング}\label{csvux30c7ux30fcux30bfux306eux30b9ux30b1ux30fcux30eaux30f3ux30b0}}

\begin{itemize}
\tightlist
\item
  TFで上手く平均0、分散1に統一したい
\item
  tf.io.decode\_csv(LINE, record\_defaults={[}\ldots{]})

  \begin{itemize}
  \tightlist
  \item
    第1引数:CSVの1行
  \item
    第2引数:1行の列数とデータ型がわかる初期値を格納した配列

    \begin{itemize}
    \tightlist
    \item
      敢えて0.を入れると欠損時に例外を吐く
    \end{itemize}
  \item
    1次元テンソル{[}{[}\ldots{]}, \ldots, {[}\ldots{]}{]}を返す

    \begin{itemize}
    \tightlist
    \item
      tf.stack(fields{[}スライス表現{]})で{[}\ldots, \ldots,
      \ldots{]}に直す
    \end{itemize}
  \end{itemize}
\end{itemize}

\hypertarget{ux3064ux306bux307eux3068ux3081ux308b}{%
\subsection{13.1.4
1つにまとめる}\label{ux3064ux306bux307eux3068ux3081ux308b}}

\hypertarget{ux3053ux308cux307eux3067ux306eux51e6ux7406ux3092ux95a2ux6570ux5316}{%
\subsubsection{これまでの処理を関数化}\label{ux3053ux308cux307eux3067ux306eux51e6ux7406ux3092ux95a2ux6570ux5316}}

\begin{itemize}
\tightlist
\item
  新要素は最後の行のprefetch(1)のみ(後述)
\end{itemize}

\hypertarget{ux30d7ux30eaux30d5ux30a7ux30c3ux30c1}{%
\subsection{13.1.5
プリフェッチ}\label{ux30d7ux30eaux30d5ux30a7ux30c3ux30c1}}

\hypertarget{ux30d7ux30eaux30d5ux30a7ux30c3ux30c1ux306bux3088ux308bux4e26ux5217ux5316}{%
\subsubsection{プリフェッチによる並列化}\label{ux30d7ux30eaux30d5ux30a7ux30c3ux30c1ux306bux3088ux308bux4e26ux5217ux5316}}

\begin{itemize}
\tightlist
\item
  prefetch(n)

  \begin{itemize}
  \tightlist
  \item
    n個のバッチをCPUとGPUで並列・高速化
  \item
    一般にn=1でよい
  \item
    GPUのRAMの容量や帯域幅(速度)が重要
  \item
    引数num\_parallel\_calls=tf.data.experimental.AUTOTUNE

    \begin{itemize}
    \tightlist
    \item
      CPU内で並列化
    \item
      nを自動調節
    \end{itemize}
  \end{itemize}
\end{itemize}

\hypertarget{ux305dux306eux4ed6ux306eux4fbfux5229ux306aux95a2ux6570}{%
\subsubsection{その他の便利な関数}\label{ux305dux306eux4ed6ux306eux4fbfux5229ux306aux95a2ux6570}}

\begin{itemize}
\tightlist
\item
  cache()

  \begin{itemize}
  \tightlist
  \item
    データセットがメモリに入る程度に小さいときに高速化
  \item
    ロード前処理とシャッフルの間で実行
  \end{itemize}
\item
  concatenate()
\item
  zip()
\item
  window()
\item
  reduce()
\item
  shard()(破片)
\item
  flat\_map()
\item
  padded\_batch()
\item
  from\_generator()
\item
  from\_tensors()
\end{itemize}

\hypertarget{tf.kerasux306eux3082ux3068ux3067ux306eux30c7ux30fcux30bfux30bbux30c3ux30c8ux306eux4f7fux3044ux65b9}{%
\subsection{13.1.6
tf.kerasのもとでのデータセットの使い方}\label{tf.kerasux306eux3082ux3068ux3067ux306eux30c7ux30fcux30bfux30bbux30c3ux30c8ux306eux4f7fux3044ux65b9}}

\hypertarget{ux30c7ux30fcux30bfux30bbux30c3ux30c8ux306eux5fdcux7528}{%
\subsubsection{データセットの応用}\label{ux30c7ux30fcux30bfux30bbux30c3ux30c8ux306eux5fdcux7528}}

\begin{itemize}
\tightlist
\item
  種々のデータセットの作成
\item
  モデルの構築と訓練
\item
  テストの評価と新インスタンスの予測
\end{itemize}

\hypertarget{ux72ecux81eaux306etfux8a13ux7df4ux95a2ux657012ux7ae0ux3068ux540cux69d8}{%
\subsubsection{独自のTF訓練関数(12章と同様)}\label{ux72ecux81eaux306etfux8a13ux7df4ux95a2ux657012ux7ae0ux3068ux540cux69d8}}

\begin{itemize}
\tightlist
\item
  p,401の自動微分のコードと比較するとよい
\end{itemize}

\hypertarget{tfrecordux5f62ux5f0f}{%
\section{13.2 TFRecord形式}\label{tfrecordux5f62ux5f0f}}

\hypertarget{tfrecord-ux6982ux8981}{%
\subsubsection{TFRecord 概要}\label{tfrecord-ux6982ux8981}}

\begin{itemize}
\tightlist
\item
  データのロードやパースがボトルネックなら用いる
\item
  大規模なデータの効率的な格納・読み出しが可能
\item
  可変長バイナリレコードのシーケンス

  \begin{itemize}
  \tightlist
  \item
    長さ情報
  \item
    長さ情報のCRCチェックサム
  \item
    実データ
  \item
    実データのチェックサム
  \end{itemize}
\end{itemize}

\hypertarget{tfrecordux306eux5229ux7528}{%
\subsubsection{TFRecordの利用}\label{tfrecordux306eux5229ux7528}}

\begin{itemize}
\tightlist
\item
  書き込み
\item
  読み出し
\item
  出力
\end{itemize}

\hypertarget{tfrecordux30d5ux30a1ux30a4ux30ebux306eux5727ux7e2e}{%
\subsection{13.2.1
TFRecordファイルの圧縮}\label{tfrecordux30d5ux30a1ux30a4ux30ebux306eux5727ux7e2e}}

\hypertarget{tfrecordux30d5ux30a1ux30a4ux30ebux306eux5727ux7e2e-1}{%
\subsubsection{TFRecordファイルの圧縮}\label{tfrecordux30d5ux30a1ux30a4ux30ebux306eux5727ux7e2e-1}}

\begin{itemize}
\tightlist
\item
  送受信に便利
\item
  option引数で指定
\item
  解凍
\end{itemize}

\hypertarget{ux30d7ux30edux30c8ux30b3ux30ebux30d0ux30c3ux30d5ux30a1ux5165ux9580}{%
\subsection{13.2.2
プロトコルバッファ入門}\label{ux30d7ux30edux30c8ux30b3ux30ebux30d0ux30c3ux30d5ux30a1ux5165ux9580}}

\hypertarget{protobufux30d7ux30edux30c8ux30b3ux30ebux30d0ux30c3ux30d5ux30a1}{%
\subsubsection{protobuf(プロトコルバッファ)}\label{protobufux30d7ux30edux30c8ux30b3ux30ebux30d0ux30c3ux30d5ux30a1}}

\begin{itemize}
\tightlist
\item
  通常のTFRecordで使われるシリアライズ化されたバイナリ形式
\item
  可搬性と拡張性に優れる
\item
  protoc:protobufコンパイラ
\item
  .protocファイル内の定義例
\end{itemize}

\hypertarget{pythonux306eprotobufux30a2ux30afux30bbux30b9ux30afux30e9ux30b9ux306eux4f7fux7528ux4f8b}{%
\subsubsection{Pythonのprotobufアクセスクラスの使用例}\label{pythonux306eprotobufux30a2ux30afux30bbux30b9ux30afux30e9ux30b9ux306eux4f7fux7528ux4f8b}}

\begin{itemize}
\tightlist
\item
  personインスタンスの可視化や読み書き
\item
  SerializeToString()でシリアライズ
\item
  ParseFromString()でデシリアライズ
\end{itemize}

\hypertarget{protobufux3068tfux3068ux306eux95a2ux4fc2}{%
\subsubsection{protobufとTFとの関係}\label{protobufux3068tfux3068ux306eux95a2ux4fc2}}

\begin{itemize}
\tightlist
\item
  protocファイルで定義したPersonクラスはTFオペレーションでない

  \begin{itemize}
  \tightlist
  \item
    TF関数に単純に組み込めない
  \item
    tf.py\_function()で組み込むと速度と可搬性が落ちる
  \item
    TFで定義された特別なprotobuf定義で解決する
  \end{itemize}
\item
  その他詳細

  \begin{itemize}
  \tightlist
  \item
    https://developers.google.com/protocol-buffers/
  \end{itemize}
\end{itemize}
