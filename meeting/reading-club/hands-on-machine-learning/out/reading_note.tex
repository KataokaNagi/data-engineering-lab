\hypertarget{reading-note}{%
\section{Reading Note}\label{reading-note}}

\begin{itemize}
\tightlist
\item
  片岡輪読担当分

  \begin{itemize}
  \tightlist
  \item
    1.1-1.4 2021/03/17
  \item
    10前半 2021/04/07
  \item
    13前半 2021/04/28
  \item
    16前半 2021/05/26
  \end{itemize}
\end{itemize}

\hypertarget{ux306fux3058ux3081ux306b}{%
\subsection{はじめに}\label{ux306fux3058ux3081ux306b}}

\begin{itemize}
\tightlist
\item
  アプローチ

  \begin{itemize}
  \tightlist
  \item
    scikit-learn

    \begin{itemize}
    \tightlist
    \item
      多くの効率的なアルゴリズム
    \end{itemize}
  \item
    TensorFlow

    \begin{itemize}
    \tightlist
    \item
      GPUによる分散NNエンジン
    \item
      Google
    \end{itemize}
  \item
    Keras

    \begin{itemize}
    \tightlist
    \item
      NNの単純化API
    \item
      TensorFlowなどと利用
    \end{itemize}
  \end{itemize}
\item
  必要な予備知識

  \begin{itemize}
  \tightlist
  \item
    Python

    \begin{itemize}
    \tightlist
    \item
      Numpy
    \item
      Pandas
    \item
      Matplotlib
    \end{itemize}
  \item
    数学

    \begin{itemize}
    \tightlist
    \item
      解析学
    \item
      線形代数
    \item
      確率論
    \item
      統計学
    \end{itemize}
  \end{itemize}
\item
  おすすめ教材

  \begin{itemize}
  \tightlist
  \item
    Pythonチュートリアル

    \begin{itemize}
    \tightlist
    \item
      \href{https://www.learnpython.org/}{LearnPython}
    \item
      \href{https://docs.python.org/3/tutorial/}{python.org}
    \end{itemize}
  \item
    機械学習

    \begin{itemize}
    \tightlist
    \item
      \href{https://www.coursera.org/learn/machine-learning}{Cousera -
      Andrew Ng 機械学習講座}

      \begin{itemize}
      \tightlist
      \item
        数か月かかる
      \end{itemize}
    \item
      \href{https://scikit-learn.org/skdoc}{scikit-learn ユーザーガイド}
    \item
      \href{https://www.dataquest.io/}{Dataquest}

      \begin{itemize}
      \tightlist
      \item
        対話的教材
      \end{itemize}
    \item
      \href{https://jp.quora.com/search?q=Machine\%20Learning}{Quora}

      \begin{itemize}
      \tightlist
      \item
        Q\&Aサイト
      \end{itemize}
    \item
      \href{https://sites.google.com/site/minggaoshomepage/links/dee}{deeplearning.net}
    \end{itemize}
  \end{itemize}
\item
  配布コード

  \begin{itemize}
  \tightlist
  \item
    Jyoyterノートブック
  \item
    https://github.com/ageron/handson-ml2
  \end{itemize}
\end{itemize}

\hypertarget{i-ux6a5fux68b0ux5b66ux7fd2ux306eux57faux790e}{%
\subsection{I
機械学習の基礎}\label{i-ux6a5fux68b0ux5b66ux7fd2ux306eux57faux790e}}

\hypertarget{ux6a5fux68b0ux5b66ux7fd2ux306eux73feux72b6}{%
\section{1
機械学習の現状}\label{ux6a5fux68b0ux5b66ux7fd2ux306eux73feux72b6}}

\begin{itemize}
\tightlist
\item
  MLの対象や意味範囲
\item
  MLの例

  \begin{itemize}
  \tightlist
  \item
    スパムフィルタ
  \item
    OCR
  \item
    商品提案
  \item
    音声検索
  \end{itemize}
\end{itemize}

\hypertarget{ux6a5fux68b0ux5b66ux7fd2ux3068ux306fux4f55ux304b}{%
\subsection{1.1
機械学習とは何か}\label{ux6a5fux68b0ux5b66ux7fd2ux3068ux306fux4f55ux304b}}

\begin{itemize}
\tightlist
\item
  コンピュータがデータから学習するための科学技術

  \begin{itemize}
  \tightlist
  \item
    学習:タスクの測定指標が向上する経験を得ること
  \end{itemize}
\item
  語義

  \begin{itemize}
  \tightlist
  \item
    訓練セット

    \begin{itemize}
    \tightlist
    \item
      学習用のデータ例
    \end{itemize}
  \item
    訓練インスタンス,標本

    \begin{itemize}
    \tightlist
    \item
      個々のデータ例
    \end{itemize}
  \item
    訓練データ

    \begin{itemize}
    \tightlist
    \item
      経験
    \end{itemize}
  \item
    正解率

    \begin{itemize}
    \tightlist
    \item
      性能指標
    \end{itemize}
  \end{itemize}
\end{itemize}

\hypertarget{ux306aux305cux6a5fux68b0ux5b66ux7fd2ux3092ux4f7fux3046ux306eux304b}{%
\subsection{1.2
なぜ機械学習を使うのか}\label{ux306aux305cux6a5fux68b0ux5b66ux7fd2ux3092ux4f7fux3046ux306eux304b}}

\begin{itemize}
\tightlist
\item
  MLの手順

  \begin{enumerate}
  \def\labelenumi{\arabic{enumi}.}
  \tightlist
  \item
    一般的な特徴を分析
  \item
    特徴をもとに検出アルゴリズムを作成
  \item
    プログラムをテストし、実用レベルになるまで上を繰り返す
  \end{enumerate}
\item
  MLの利点

  \begin{itemize}
  \tightlist
  \item
    データごとのアルゴリズムが不要
  \item
    複雑な問題の解決
  \item
    既知のアルゴリズムがない問題の解決
  \item
    データマイニング

    \begin{itemize}
    \tightlist
    \item
      特徴の予想外な相関関係,トレンドの発見

      \begin{itemize}
      \tightlist
      \item
        高速に
      \end{itemize}
    \end{itemize}
  \end{itemize}
\end{itemize}

\hypertarget{ux5fdcux7528ux306eux4f8b}{%
\subsection{1.3 応用の例}\label{ux5fdcux7528ux306eux4f8b}}

\begin{itemize}
\tightlist
\item
  製品の自動分類

  \begin{itemize}
  \tightlist
  \item
    イメージ分類

    \begin{itemize}
    \tightlist
    \item
      CNN(Convolutional:畳み込み)などを利用
    \end{itemize}
  \end{itemize}
\item
  脳腫瘍の検出

  \begin{itemize}
  \tightlist
  \item
    セマンティックセグメンテーション

    \begin{itemize}
    \tightlist
    \item
      CNNなど
    \end{itemize}
  \end{itemize}
\item
  記事の自動分類

  \begin{itemize}
  \tightlist
  \item
    テキスト分類 ∈ NLP(自然言語処理)

    \begin{itemize}
    \tightlist
    \item
      RNN(Recurrent:再帰型)
    \item
      CNN
    \item
      Transformer
    \end{itemize}
  \end{itemize}
\item
  不適切発言へのフラグ付加

  \begin{itemize}
  \tightlist
  \item
    テキスト分類
  \end{itemize}
\item
  自動要約

  \begin{itemize}
  \tightlist
  \item
    テキスト自動要約 ∈ NLP
  \end{itemize}
\item
  チャットボット、パーソナルアシスタント

  \begin{itemize}
  \tightlist
  \item
    NLU(自然言語理解)
  \item
    Q\&Aモジュール
  \end{itemize}
\item
  次年度収益の予測

  \begin{itemize}
  \tightlist
  \item
    回帰タスク(値の予測)

    \begin{itemize}
    \tightlist
    \item
      線形回帰
    \item
      多項式回帰モデル
    \item
      SVM回帰
    \item
      ランダムフォレスト回帰
    \item
      人工NN
    \end{itemize}
  \item
    過去の業績指標の利用

    \begin{itemize}
    \tightlist
    \item
      RNN
    \item
      CNN
    \item
      Transformer
    \end{itemize}
  \end{itemize}
\item
  音声コマンド

  \begin{itemize}
  \tightlist
  \item
    オーディオサンプルの処理

    \begin{itemize}
    \tightlist
    \item
      長くて複雑

      \begin{itemize}
      \tightlist
      \item
        RNN
      \item
        CNN
      \item
        Transformer
      \end{itemize}
    \end{itemize}
  \end{itemize}
\item
  クレカ詐欺の検知

  \begin{itemize}
  \tightlist
  \item
    異常検知
  \end{itemize}
\item
  購入履歴による顧客分類、販売戦略

  \begin{itemize}
  \tightlist
  \item
    クラスタリング
  \end{itemize}
\item
  高次元データセットの図示

  \begin{itemize}
  \tightlist
  \item
    データの可視化
  \item
    次元削除
  \end{itemize}
\item
  購入履歴から商品提案

  \begin{itemize}
  \tightlist
  \item
    推薦システム

    \begin{itemize}
    \tightlist
    \item
      人工NNなど
    \end{itemize}
  \end{itemize}
\item
  ゲームのインテリジェントボット

  \begin{itemize}
  \tightlist
  \item
    アクションの選択

    \begin{itemize}
    \tightlist
    \item
      RL(Reinforced L:強化学習)

      \begin{itemize}
      \tightlist
      \item
        全時間の報酬の最大化
      \item
        AlphaGo
      \end{itemize}
    \end{itemize}
  \end{itemize}
\item
  Appendix:画像系 by 加瀬先輩

  \begin{itemize}
  \tightlist
  \item
    CNNs:

    \begin{itemize}
    \tightlist
    \item
      Convolutional Neural
      Networks、種類)AlexNet、VGG16、ResNet、GoogLeNet
    \end{itemize}
  \item
    Feature:

    \begin{itemize}
    \tightlist
    \item
      特徴、CNNの抽出層で返還されたデータのことや、画像に内在するパターンのこと
      類)Feature map
    \end{itemize}
  \item
    Image Classification:

    \begin{itemize}
    \tightlist
    \item
      画像分類タスク.
    \end{itemize}
  \item
    Image Recognition:

    \begin{itemize}
    \tightlist
    \item
      画像認識タスク、Detectionと似ているかも?どの範囲に物体が存在するか検知
    \end{itemize}
  \item
    Image Generation:

    \begin{itemize}
    \tightlist
    \item
      画像生成、類)GAN
    \end{itemize}
  \item
    Image Segmentation:

    \begin{itemize}
    \tightlist
    \item
      画像分割、あるルールに則ってパーツごとに分割するタスク
    \end{itemize}
  \item
    MNIST:

    \begin{itemize}
    \tightlist
    \item
      0\textasciitilde9の手書きの数字(白黒)データセット
    \end{itemize}
  \item
    CIFAR10:

    \begin{itemize}
    \tightlist
    \item
      一般物体認識用のデータセット、飛行機・船・自動車・トラック・猫・犬・蛙・馬・鳥・鹿の10クラス
    \end{itemize}
  \item
    ImageNet:

    \begin{itemize}
    \tightlist
    \item
      画像分類で評価に使用されるデータセット
    \end{itemize}
  \item
    GAN:

    \begin{itemize}
    \tightlist
    \item
      Generative Adversarial Networks、画像を生成するモデル.
      BLEACH風(KBTIT)の顔変換等のpaperは面白い.
    \end{itemize}
  \item
    Adversarial Examples:

    \begin{itemize}
    \tightlist
    \item
      敵対的サンプル、AIを騙す画像.
    \end{itemize}
  \item
    XAI:

    \begin{itemize}
    \tightlist
    \item
      説明可能AI、最近の流行り. モデルの予測根拠を可視化など.
      ex)SHAP、LIME、SmoothGrad、Guided-BackProp etc
    \end{itemize}
  \item
    Robustness:

    \begin{itemize}
    \tightlist
    \item
      頑健性、研究フィールドの文脈で意味合いが異なるが、ノイズなどの影響を受けずに正しく分類する性質
    \end{itemize}
  \item
    Generalization:

    \begin{itemize}
    \tightlist
    \item
      汎化、未知のデータに対しても正しく分類できる性質
    \end{itemize}
  \item
    Data Distribution:

    \begin{itemize}
    \tightlist
    \item
      データ分布(画像に限らず)、データセット1枚1枚で見た時に、青い鳥が多いとか、犬の背景は山が多いとか傾向
    \end{itemize}
  \item
    Label annotation:

    \begin{itemize}
    \tightlist
    \item
      画像に付与された正解ラベルは本当に適切か?(複数写っている場合はどうする
      etc)という問題に対応する学問
    \end{itemize}
  \item
    Bouding Box:

    \begin{itemize}
    \tightlist
    \item
      認識タスクで、物体が映り込んでいる領域のこと.
      物体に合わせて適切に囲むことが目的
    \end{itemize}
  \end{itemize}
\end{itemize}

\hypertarget{ux6a5fux68b0ux5b66ux7fd2ux30b7ux30b9ux30c6ux30e0ux306eux30bfux30a4ux30d7}{%
\subsection{1.4
機械学習システムのタイプ}\label{ux6a5fux68b0ux5b66ux7fd2ux30b7ux30b9ux30c6ux30e0ux306eux30bfux30a4ux30d7}}

\begin{itemize}
\tightlist
\item
  MLの分類

  \begin{itemize}
  \tightlist
  \item
    人間の関与の有無

    \begin{itemize}
    \tightlist
    \item
      教師あり
    \item
      教師なし
    \item
      半教師あり
    \item
      強化学習
    \end{itemize}
  \item
    その場で少しずつ学習可能か

    \begin{itemize}
    \tightlist
    \item
      オンライン
    \item
      バッチ学習
    \end{itemize}
  \item
    既知のデータポイントから予測モデルを構築するか

    \begin{itemize}
    \tightlist
    \item
      インスタンスベース学習
    \item
      モデルベース学習
    \end{itemize}
  \end{itemize}
\item
  上記分類を組み合わせる
\end{itemize}

\hypertarget{ux6559ux5e2bux3042ux308aux6559ux5e2bux306aux3057ux5b66ux7fd2}{%
\section{1.4.1
教師あり/教師なし学習}\label{ux6559ux5e2bux3042ux308aux6559ux5e2bux306aux3057ux5b66ux7fd2}}

\begin{itemize}
\tightlist
\item
  教師あり学習

  \begin{itemize}
  \tightlist
  \item
    訓練データに正解のラベル
  \item
    応用先

    \begin{itemize}
    \tightlist
    \item
      クラス分類
    \item
      回帰(regression)

      \begin{itemize}
      \tightlist
      \item
        予測子(predictor)(=一連の特徴量)からターゲットの数値を予測すること
      \item
        語源:背の高い両親の子供が平均に帰して背が低くなる傾向にあることを統計的に示した研究より
      \item
        出力が複数になる問題も
      \end{itemize}
    \item
      回帰に分類が使えるものも
    \item
      分類に回帰が使えるものも

      \begin{itemize}
      \tightlist
      \item
        例:ロジスティック回帰で分類確率を導出
      \end{itemize}
    \end{itemize}
  \item
    代表的なアルゴリズム

    \begin{itemize}
    \tightlist
    \item
      k近傍法
    \item
      線形回帰
    \item
      ロジスティック回帰
    \item
      SVM
    \item
      決定木,ランダムフォレスト
    \item
      NN
    \end{itemize}
  \end{itemize}
\item
  教師なし学習

  \begin{itemize}
  \tightlist
  \item
    ラベルはない
  \item
    代表的なアルゴリズム

    \begin{itemize}
    \tightlist
    \item
      クラスタリング

      \begin{itemize}
      \tightlist
      \item
        例

        \begin{itemize}
        \tightlist
        \item
          k平均法
        \item
          DBSCAN
        \item
          階層的クラスタ分析(HCA:Hierarchial Clustering Analysis?)

          \begin{itemize}
          \tightlist
          \item
            分類の粒度が可変
          \end{itemize}
        \end{itemize}
      \end{itemize}
    \item
      異常検知、新規性検知(anomaly detection, novelty detection)

      \begin{itemize}
      \tightlist
      \item
        異常検知

        \begin{itemize}
        \tightlist
        \item
          正常なインスタンスで訓練
        \item
          想定内の外れ値を検知
        \end{itemize}
      \item
        新規性検知

        \begin{itemize}
        \tightlist
        \item
          ML以外のアルゴリズムで検知できない、分類済みだと思われるインスタンスで訓練
        \item
          予想外の外れ値を検知
        \end{itemize}
      \item
        例

        \begin{itemize}
        \tightlist
        \item
          1クラスSVM
        \item
          アイソレーションフォレスト
        \end{itemize}
      \end{itemize}
    \item
      可視化、次元削減(visualization, dimension reduction)

      \begin{itemize}
      \tightlist
      \item
        構造を保ちつつ、セマンティッククラスタを可視化

        \begin{itemize}
        \tightlist
        \item
          片岡解釈:クラスタごとの座標などを保ちつつ、意味のある群の可視化用ラベル(次元)を増やす
        \item
          可視化例:クラスタ同士の重なり具合
        \end{itemize}
      \item
        特徴量抽出(Feature Extraction)

        \begin{itemize}
        \tightlist
        \item
          情報量を保ちつつ相関する複数の特徴をまとめ(次元圧縮)、データを見やすくする
        \item
          種々のMLの前処理に利用

          \begin{itemize}
          \tightlist
          \item
            速度向上
          \item
            計算資源の節約
          \item
            性能が向上することも
          \end{itemize}
        \end{itemize}
      \item
        例

        \begin{itemize}
        \tightlist
        \item
          PCA(Principal Component Analysis:主成分分析)
        \item
          カーネルPCA
        \item
          LLE(Locally-Linear Embedding:局所線形埋め込み法)
        \item
          t-SNE(t-distributed Stochastic Neighbor
          Embedding:t分布確率的近傍埋め込み法)
        \end{itemize}
      \end{itemize}
    \item
      相関ルール学習(association rule learning)

      \begin{itemize}
      \tightlist
      \item
        大量のデータの属性同士の興味深い関係を導く

        \begin{itemize}
        \tightlist
        \item
          例:オムツとビール
        \end{itemize}
      \item
        例

        \begin{itemize}
        \tightlist
        \item
          ア・プリオリ
        \item
          Eclat
        \end{itemize}
      \end{itemize}
    \end{itemize}
  \end{itemize}
\item
  半教師あり学習(semisupervised learning)

  \begin{itemize}
  \tightlist
  \item
    一部にラベル
  \item
    ラベルの有無に応じた重みづけ?
  \item
    一部にラベリングして全体にラベリングすることも
  \item
    教師あり・なしの同時使用が多い
  \item
    例

    \begin{itemize}
    \tightlist
    \item
      DBN(deep brief network)

      \begin{itemize}
      \tightlist
      \item
        教師なしのRBM(restricted Boltzmann machines:制限付きボルツマン
      \item
        マシン)
      \item
        教師ありで微調整
      \end{itemize}
    \end{itemize}
  \end{itemize}
\item
  強化学習

  \begin{itemize}
  \tightlist
  \item
    流れ

    \begin{enumerate}
    \def\labelenumi{\arabic{enumi}.}
    \tightlist
    \item
      エージェント(学習システム)が
    \item
      環境を観察し
    \item
      行動を選択して実行し
    \item
      報酬 or ペナルティを得る。
    \item
      報酬の高い方策(policy)を学習する。
    \item
      学習した方策に従い、特別な行動を決定う
    \end{enumerate}
  \item
    例

    \begin{itemize}
    \tightlist
    \item
      ロボットの歩行
    \item
      AlphaGo

      \begin{itemize}
      \tightlist
      \item
        自分自身とも対局
      \end{itemize}
    \end{itemize}
  \end{itemize}
\end{itemize}

\hypertarget{ux30d0ux30c3ux30c1ux5b66ux7fd2ux3068ux30aaux30f3ux30e9ux30a4ux30f3ux5b66ux7fd2}{%
\section{1.4.2
バッチ学習とオンライン学習}\label{ux30d0ux30c3ux30c1ux5b66ux7fd2ux3068ux30aaux30f3ux30e9ux30a4ux30f3ux5b66ux7fd2}}

\begin{itemize}
\tightlist
\item
  バッチ学習(オフライン学習)

  \begin{itemize}
  \tightlist
  \item
    事前に全ての訓練データで学習
  \item
    オフラインで大量の時間と計算資源を割く

    \begin{itemize}
    \tightlist
    \item
      流動的なデータには不向き
    \item
      分割して学習できないために計算資源に余裕が必要
    \end{itemize}
  \end{itemize}
\item
  オンライン学習(差分学習 incremental L.)

  \begin{itemize}
  \tightlist
  \item
    ミニバッチで段階的に訓練
  \item
    使用済みデータの保持が不要

    \begin{itemize}
    \tightlist
    \item
      アウトオブコア(主記憶より大容量の学習システム)でも使用可能

      \begin{itemize}
      \tightlist
      \item
        通常\textbf{オフ}ラインで使用
      \end{itemize}
    \end{itemize}
  \item
    学習速度(L. rate)が重要

    \begin{itemize}
    \tightlist
    \item
      速いと古いデータを忘れる
    \item
      遅いと外れ値に強くなる

      \begin{itemize}
      \tightlist
      \item
        実用上、速さは大事?
      \end{itemize}
    \end{itemize}
  \item
    性能が低下するデータの学習は打ち止める

    \begin{itemize}
    \tightlist
    \item
      異常検知などを併用
    \end{itemize}
  \end{itemize}
\end{itemize}

\hypertarget{ux30a4ux30f3ux30b9ux30bfux30f3ux30b9ux30d9ux30fcux30b9ux5b66ux7fd2ux3068ux30e2ux30c7ux30ebux30d9ux30fcux30b9ux5b66ux7fd2}{%
\section{1.4.3
インスタンスベース学習とモデルベース学習}\label{ux30a4ux30f3ux30b9ux30bfux30f3ux30b9ux30d9ux30fcux30b9ux5b66ux7fd2ux3068ux30e2ux30c7ux30ebux30d9ux30fcux30b9ux5b66ux7fd2}}

\begin{itemize}
\tightlist
\item
  汎化(generalize)によるMLの分類

  \begin{itemize}
  \tightlist
  \item
    新しいデータの予測のためのアプローチ
  \end{itemize}
\item
  インスタンスベース学習

  \begin{itemize}
  \tightlist
  \item
    既存データの丸暗記
  \item
    新しいデータに類似度の尺度(measure of similarity)を適用
  \item
    例

    \begin{itemize}
    \tightlist
    \item
      k近傍法
    \end{itemize}
  \end{itemize}
\item
  モデルベース学習

  \begin{itemize}
  \tightlist
  \item
    線形関数、超平面などにモデリング

    \begin{itemize}
    \tightlist
    \item
      モデルパラメータΘ\_i(慣例)による関数f(Θ\_i)(モデル)を定義

      \begin{itemize}
      \tightlist
      \item
        モデルを評価する関数を定義

        \begin{itemize}
        \tightlist
        \item
          良さを示す適応度関数g(f(Θ\_i))?(utility f.)
        \item
          悪さを示すコスト関数h(f(Θ\_i))?
        \end{itemize}
      \item
        Θ\_iを変化させてg or hを大きくor小さくさせるよにに訓練(not
        学習?)
      \end{itemize}
    \end{itemize}
  \item
    3種類の「モデル」に注意

    \begin{itemize}
    \tightlist
    \item
      線形回帰などのモデル
    \item
      線形回帰などから構築したモデルアーキテクチャ
    \item
      訓練済みのモデル
    \end{itemize}
  \item
    \href{https://github.com/ageron/handson-ml2/blob/master/01_the_machine_learning_landscape.ipynb}{線形モデルでGDPと暮らし満足度の予測を行うコード}

    \begin{itemize}
    \tightlist
    \item
      モデルベースもインスタンスベースも近い値を取る
    \item
      必要操作

      \begin{itemize}
      \tightlist
      \item
        データの検討
      \item
        モデルの選択
      \item
        コスト関数による訓練
      \item
        新しいデータで推論(inference) ``` \# Code example import
        matplotlib.pyplot as plt import numpy as np import pandas as pd
        import sklearn.linear\_model \# import sklearn.neighbors
      \end{itemize}
    \end{itemize}
  \end{itemize}
\end{itemize}

\hypertarget{load-the-data}{%
\section{Load the data}\label{load-the-data}}

oecd\_bli = pd.read\_csv(datapath + ``oecd\_bli\_2015.csv'',
thousands=`,') gdp\_per\_capita = pd.read\_csv(datapath +
``gdp\_per\_capita.csv'',thousands=`,',delimiter=`\t',
encoding='latin1', na\_values=``n/a'')

\hypertarget{prepare-the-data}{%
\section{Prepare the data}\label{prepare-the-data}}

country\_stats = prepare\_country\_stats(oecd\_bli, gdp\_per\_capita) X
= np.c\_{[}country\_stats{[}``GDP per capita''{]}{]} y =
np.c\_{[}country\_stats{[}``Life satisfaction''{]}{]}

\hypertarget{visualize-the-data}{%
\section{Visualize the data}\label{visualize-the-data}}

country\_stats.plot(kind=`scatter', x=``GDP per capita'', y=`Life
satisfaction') plt.show()

\hypertarget{select-a-linear-model}{%
\section{Select a linear model}\label{select-a-linear-model}}

model = sklearn.linear\_model.LinearRegression() \# model =
sklearn.neighbors.KNeighborsRegressor(n\_neighbors=3)

\hypertarget{train-the-model}{%
\section{Train the model}\label{train-the-model}}

model.fit(X, y)

\hypertarget{make-a-prediction-for-cyprus}{%
\section{Make a prediction for
Cyprus}\label{make-a-prediction-for-cyprus}}

X\_new = {[}{[}22587{]}{]} \# Cyprus' GDP per capita
print(model.predict(X\_new)) \# outputs {[}{[} 5.96242338{]}{]} ```

\hypertarget{ux6a5fux68b0ux5b66ux7fd2ux304cux62b1ux3048ux308bux96e3ux554f}{%
\subsection{1.5
機械学習が抱える難問}\label{ux6a5fux68b0ux5b66ux7fd2ux304cux62b1ux3048ux308bux96e3ux554f}}

\hypertarget{ux8a13ux7df4ux30c7ux30fcux30bfux4f8bux306eux54c1ux8ceaux306eux4f4eux3055}{%
\subsection{\# 1.5.1
訓練データ例の品質の低さ}\label{ux8a13ux7df4ux30c7ux30fcux30bfux4f8bux306eux54c1ux8ceaux306eux4f4eux3055}}

\hypertarget{ux73feux5b9fux3092ux4ee3ux8868ux3057ux3066ux3044ux308bux3068ux306fux8a00ux3048ux306aux3044ux8a13ux7df4ux30c7ux30fcux30bf}{%
\subsection{\# 1.5.2
現実を代表しているとは言えない訓練データ}\label{ux73feux5b9fux3092ux4ee3ux8868ux3057ux3066ux3044ux308bux3068ux306fux8a00ux3048ux306aux3044ux8a13ux7df4ux30c7ux30fcux30bf}}

\hypertarget{ux54c1ux8ceaux306eux4f4eux3044ux30c7ux30fcux30bf}{%
\subsection{\# 1.5.3
品質の低いデータ}\label{ux54c1ux8ceaux306eux4f4eux3044ux30c7ux30fcux30bf}}

\begin{itemize}
\tightlist
\item
\item
\end{itemize}

\hypertarget{ux7121ux95a2ux4fc2ux306aux7279ux5fb4ux91cf}{%
\subsection{\# 1.5.4
無関係な特徴量}\label{ux7121ux95a2ux4fc2ux306aux7279ux5fb4ux91cf}}

\hypertarget{ux8a13ux7df4ux30c7ux30fcux30bfux3078ux306eux904eux5b66ux7fd2}{%
\subsection{\# 1.5.5
訓練データへの過学習}\label{ux8a13ux7df4ux30c7ux30fcux30bfux3078ux306eux904eux5b66ux7fd2}}

\hypertarget{ux8a13ux7df4ux30c7ux30fcux30bfux3078ux306eux904eux5c11ux9069ux5408}{%
\subsection{\# 1.5.6
訓練データへの過少適合}\label{ux8a13ux7df4ux30c7ux30fcux30bfux3078ux306eux904eux5c11ux9069ux5408}}

\hypertarget{ux4e00ux6b69ux4e0bux304cux3063ux3066ux5fa9ux7fd2ux3057ux3088ux3046}{%
\subsection{\# 1.5.7
一歩下がって復習しよう}\label{ux4e00ux6b69ux4e0bux304cux3063ux3066ux5fa9ux7fd2ux3057ux3088ux3046}}

\hypertarget{ux30c6ux30b9ux30c8ux3068ux691cux8a3c}{%
\subsection{1.6
テストと検証}\label{ux30c6ux30b9ux30c8ux3068ux691cux8a3c}}

\hypertarget{ux30cfux30a4ux30d1ux30fcux30d1ux30e9ux30e1ux30fcux30bfux306eux8abfux6574ux3068ux30e2ux30c7ux30ebux306eux9078ux629e}{%
\subsection{\# 1.6.1
ハイパーパラメータの調整とモデルの選択}\label{ux30cfux30a4ux30d1ux30fcux30d1ux30e9ux30e1ux30fcux30bfux306eux8abfux6574ux3068ux30e2ux30c7ux30ebux306eux9078ux629e}}

\hypertarget{ux30c7ux30fcux30bfux306eux30dfux30b9ux30deux30c3ux30c1}{%
\subsection{\# 1.6.2
データのミスマッチ}\label{ux30c7ux30fcux30bfux306eux30dfux30b9ux30deux30c3ux30c1}}

\hypertarget{ux6f14ux7fd2ux554fux984c}{%
\subsection{1.7 演習問題}\label{ux6f14ux7fd2ux554fux984c}}

\begin{itemize}
\tightlist
\item
  片岡担当分(1.4)まで

  \begin{enumerate}
  \def\labelenumi{\arabic{enumi}.}
  \tightlist
  \item
    機械学習の定義

    \begin{itemize}
    \tightlist
    \item
      片岡の解答

      \begin{itemize}
      \tightlist
      \item
        コンピュータが、データからタスクの測定指標が向上する経験を得るための科学技術
      \end{itemize}
    \item
      テキストの解答

      \begin{itemize}
      \tightlist
      \item
        データから学習できるシステムをつくること

        \begin{itemize}
        \tightlist
        \item
          学習:何らかの測定手段に基づき、あるタスクを処理した成績が上がる操作
        \end{itemize}
      \end{itemize}
    \end{itemize}
  \item
    機械学習が発揮する問題の4タイプ

    \begin{itemize}
    \tightlist
    \item
      片岡の解答

      \begin{itemize}
      \tightlist
      \item
        データごとのアルゴリズム実装が大変な問題
      \item
        複雑な問題
      \item
        既知のアルゴリズムがない問題
      \item
        予想外な相関関係とトレンドを抽出する問題(データマイニング)
      \end{itemize}
    \item
      テキストの解答

      \begin{itemize}
      \tightlist
      \item
        アルゴリズムを使ったソリューションがない複雑な問題の解決
      \item
        思い付きの規則が延々と続くものに代わるモジュールの開発
      \item
        変動する環境に合わせて自分を修正できるシステムの開発
      \item
        人間の学習の支援(データマイニングなど)
      \end{itemize}
    \end{itemize}
  \item
    ラベル付き訓練セットとは

    \begin{itemize}
    \tightlist
    \item
      片岡の解答

      \begin{itemize}
      \tightlist
      \item
        人間による分類を行った全ての訓練データ
      \end{itemize}
    \item
      テキストの解答

      \begin{itemize}
      \tightlist
      \item
        個々のインスタンスに問題の答えが含まれている訓練セット
      \end{itemize}
    \end{itemize}
  \item
    教師あり学習の応用例2つ

    \begin{itemize}
    \tightlist
    \item
      片岡の解答

      \begin{itemize}
      \tightlist
      \item
        回帰
      \item
        クラス分類
      \end{itemize}
    \item
      テキストの解答

      \begin{itemize}
      \tightlist
      \item
        回帰
      \item
        分類
      \end{itemize}
    \end{itemize}
  \item
    教師なし学習の応用例4つ

    \begin{itemize}
    \tightlist
    \item
      片岡の解答

      \begin{itemize}
      \tightlist
      \item
        クラスタリング
      \item
        異常検知、新規性検知
      \item
        可視化、次元削減
      \item
        相関ルール学習
      \end{itemize}
    \item
      テキストの解答

      \begin{itemize}
      \tightlist
      \item
        同上
      \end{itemize}
    \end{itemize}
  \item
    未知の領域を探索する、ロボットで使えるMLは

    \begin{itemize}
    \tightlist
    \item
      片岡の解答

      \begin{itemize}
      \tightlist
      \item
        強化学習
      \end{itemize}
    \item
      テキストの解答

      \begin{itemize}
      \tightlist
      \item
        強化学習
      \item
        教師あり・なし学習だと不自然になる
      \end{itemize}
    \end{itemize}
  \item
    顧客を分類するMLは

    \begin{itemize}
    \tightlist
    \item
      片岡の解答

      \begin{itemize}
      \tightlist
      \item
        クラスタリング(教師なし学習)
      \end{itemize}
    \item
      テキストの解答

      \begin{itemize}
      \tightlist
      \item
        集団の定義が分からない場合

        \begin{itemize}
        \tightlist
        \item
          クラスタリング(教師なし学習)
        \end{itemize}
      \item
        集団の定義が分かっている場合

        \begin{itemize}
        \tightlist
        \item
          分類アルゴリズム(教師あり学習)
        \end{itemize}
      \end{itemize}
    \end{itemize}
  \item
    スパム検出は教師ありorなしか

    \begin{itemize}
    \tightlist
    \item
      片岡の解答

      \begin{itemize}
      \tightlist
      \item
        教師あり

        \begin{itemize}
        \tightlist
        \item
          スパムの特徴は限られており、アルゴリズムを適用する必要がありそう
        \end{itemize}
      \end{itemize}
    \item
      テキストの解答

      \begin{itemize}
      \tightlist
      \item
        教師あり

        \begin{itemize}
        \tightlist
        \item
          ラベル(スパムorハム)を付けたメールでアルゴリズム訓練
        \end{itemize}
      \end{itemize}
    \end{itemize}
  \item
    オンライン学習システムとは

    \begin{itemize}
    \tightlist
    \item
      片岡の解答

      \begin{itemize}
      \tightlist
      \item
        ミニバッチで段階的に訓練する、計算資源の要求が比較的少ない学習
      \end{itemize}
    \item
      テキストの解答

      \begin{itemize}
      \tightlist
      \item
        バッチ学習システムと異なり、差分データで学習可能
      \item
        データが変化するシステムや自律的なシステムに適用可能
      \item
        機敏に学習可能
      \item
        極端に大規模なデータを使用可能
      \end{itemize}
    \end{itemize}
  \item
    アウトオブコア学習とは

    \begin{itemize}
    \tightlist
    \item
      片岡の解答

      \begin{itemize}
      \tightlist
      \item
        主記憶より大容量のデータを扱う学習システム
      \end{itemize}
    \item
      テキストの解答

      \begin{itemize}
      \tightlist
      \item
        同上
      \item
        データをミニバッチに分割し、オンライン学習のテクニックを使って学習
      \end{itemize}
    \end{itemize}
  \item
    類似度の尺度を用いる学習は

    \begin{itemize}
    \tightlist
    \item
      片岡の解答 -インスタンスベース学習
    \item
      テキストの解答

      \begin{itemize}
      \tightlist
      \item
        同上
      \item
        訓練データを丸暗記させた上で新しいインスタンスを与え、類似度の尺度から暗記したインスタンスに最も近いものを採択
      \end{itemize}
    \end{itemize}
  \end{enumerate}
\end{itemize}
